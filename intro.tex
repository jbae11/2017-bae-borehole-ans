\section{Introduction}

This work evaluates the benefits of turning a shutdown reactor,
a liability, to a borehole repository site, a much-needed facility.
This work evaluates a strategy to do so that the remaining facilities
are leveraged to cut costs and time for the repository, which makes
the repository construction and operation more economical and 
politically feasible. 

The  expected benefits of this 
proposed integrated siting strategy include reduced radioactive waste 
transportation burden, increased likelihood of consent from the local 
community, and improved expediency achieved through leveraging existing 
infrastructure and skill.

The proposed case is compared to the case of Yucca Mountain, in six
quantitative measures, in the perspective of four stakeholders.

\subsection{Motivation}
This work suggests borehole-design repositories for such integrated facility due it 
the design's modularity, wide geological suitability, and footprint 
efficiency. Borehole repositories need crystalline basement rocks at 
$ 2,000 - 5,000m$ deep, which is relatively common in the continental 
U.S \cite{arnold_research_2012}. Also, the area required for a 
borehole repository is only $30 km^2$ for the capacity of Yucca Mountain
 \cite{brady_deep_2009}.
 
The pre-existing local talent and infrastructure, is a major
benefit to the proposed design. Compared to Yucca Mountain, where little 
infrastructure existed, a shutdown power plant is more likely to have
usable infrastructure left, which can be utilized to save cost and time.

Lastly, the proposed design allows for more consent-based siting, 
since communities with nuclear facilities are more likely to be 
more receptive to the benefits of hosting a repository (i.e. jobs and taxes). 

The purpose of this paper is to attempt to quantify the benefits, and compare
them to the case of Yucca Mountain through a case study.
